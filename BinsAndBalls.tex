\section{Balls in bins, or urn problems}

You have $m$ indistinguishable balls and in front of you is a row of
$n$ bins. You place each ball into a bin chosen at random.

Let's write the sample space as $\Omega = \{1,2,\ldots, m\}^n$; in
each outcome $\omega = (\omega_1, \ldots, \omega_n)$, the value
$\omega_i$ represents the number of balls in the $i$th bin.

Here are some interesting questions one can ask.
\begin{enumerate}
\item What is the chance that the $i$th bin is empty? (What is $\pr(\omega_i = 0)$?)

Well, there are $m$ balls, and we want every single one of them to
miss the $i$th bin. The probability that the first ball misses this
bin is $(n-1)/n = 1 - 1/n$. The probability that the second ball
misses is also $1- 1/n$, as with the third, and fourth, and so
on. Therefore,
$$ \pr(\mbox{$i$th bin empty}) 
\ = \ 
\left( 1 - \frac{1}{n} \right)^m
\ \leq \ 
(e^{-1/n})^m 
\ = \ 
e^{-m/n},
$$
where we've used the formula $e^x \geq 1+x$ that we discussed earlier.

\item What is the chance that there is an empty bin if $m = n$?

Intuitively, one would expect this to be pretty close to 1 (that is,
pretty much certain), because the complementary event -- no empty bins
-- would occur only if every single bin received exactly one
ball. Let's analyze this latter probability.
$$
\pr(\mbox{every bin gets a ball}) 
\ = \ 
\left(1 - \frac{1}{n}\right) \left( 1- \frac{2}{n}\right) \cdots \frac{1}{n} 
\ = \ 
\frac{n!}{n^n}.
$$
This is miniscule; for instance, it is less than $1/2^{n/2}$.

\item What is the chance that there is an empty bin if $m = 2n\ln n$?

When $m$ is increased from $n$ to $2n \ln n$, the chance of an empty bin drops from $\approx 1$ to $\approx 0$. To see this, let $A_i$ be the event that the $i$th bin is empty. We proved above that $\pr(A_i) \leq \exp(-m/n) = \exp(-2\ln n) = 1/n^2$. Therefore
$$
\pr(\mbox{some bin is empty}) 
\ = \ 
\pr(A_1 \cup \cdots \cup A_n) 
\ \leq \ 
\pr(A_1) + \cdots + \pr(A_n)
\ \leq \ 
n \cdot \frac{1}{n^2}
\ = \ 
\frac{1}{n}
$$
(the first inequality is the union bound). Thus with high probability (at least $1-1/n$), every bin gets at least one ball.

\item {\it The coupon collector problem.} Many probability questions turn out to be thinly disguised balls-and-bins problems. Here's an example. Suppose that each cereal box contains one of $k$ action figures (chosen uniformly at random). How many cereal boxes should you buy in order to collect all $k$ figures?

If you make the following associations:
\begin{eqnarray*}
\mbox{bin} & \equiv & \mbox{action figure} \\
\mbox{throw a ball} & = & \mbox{buy a box of cereal} \\
\mbox{every bin gets a ball} & = & \mbox{you collect all the figures},
\end{eqnarray*}
then you see that what we are really asking is, how many balls do you need to throw into $n = k$ bins in order to hit all of them? And we've already seen that $m = 2k\ln k$ balls (cereal boxes) will suffice.

\item What is the probability that some bin gets two or more balls?

This is another situation in which it is easier to study the complementary event, that every bin gets at most one ball.
$$
\pr(\mbox{every bin gets $0$ or $1$ ball})
\ = \ 
\left(1 - \frac{1}{n}\right) \left( 1- \frac{2}{n} \right)\cdots \left( 1- \frac{m-1}{n} \right)
\ = \ 
\frac{n!}{(n-m)! n^m}.
$$
Although this is exact, it is a little difficult to fathom, and so let's try an approximation instead, using $1-i/n \leq e^{-i/n}$:
\begin{eqnarray*}
\pr(\mbox{every bin gets $0$ or $1$ ball})
& = & 
\left(1 - \frac{1}{n}\right) \left( 1 - \frac{2}{n} \right) \cdots \left( 1- \frac{m-1}{n} \right) \\
& \leq & \exp(-1/n) \exp(-2/n) \cdots \exp(-(m-1)/n)  \\
& = & 
\exp\left(-\frac{1}{n} \left(1 + \cdots + (m-1) \right) \right) \\
& = &  
\exp\left(-\frac{m(m-1)}{2n} \right).
\end{eqnarray*}
This is an excellent approximation when $m \ll n$. It tells us that if $m \geq c \sqrt{n}$ (for some small constant $c$), then this probability is at most $1/2$. That is, if you throw (roughly) $\sqrt{n}$ (or more) balls into $n$ bins, then chances are that some bin will get two or more balls.

\item {\it Birthday paradox.} We saw earlier than in a room with 23 people, chances are two of them will share the same birthday.

This is yet another balls-and-bins problem:
\begin{eqnarray*}
\mbox{bin} & \equiv & \mbox{day of the year} \\
\mbox{throw a ball} & \equiv & \mbox{select birthday of a person in the room} \\
\mbox{bin with $2$ balls} & \equiv & \mbox{two people with the same birthday} .
\end{eqnarray*}
The number of bins is $n = 365$ and the number of balls is $m=23$. Using the formula we just obtained,
\begin{eqnarray*}
\pr(\mbox{two people share the same birthday})
& = & 
1 - \pr(\mbox{every bin has at most one ball}) \\
& \geq & 
1 - \exp \left(-\frac{m(m-1)}{2n} \right)
\ \ \geq \ \ 
\frac{1}{2} .
\end{eqnarray*}

\item {\it Coin tossing.}

Even the tossing of a fair coin, $m$ times, can be modeled using balls and bins. Imagine there are just $n=2$ bins (call one bin $H$ and the other bin $T$).
\end{enumerate}
